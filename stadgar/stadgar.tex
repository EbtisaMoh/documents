\documentclass[]{article}
\usepackage{lmodern}
\usepackage{amssymb,amsmath}
\usepackage{ifxetex,ifluatex}
\usepackage{fixltx2e} % provides \textsubscript
\ifnum 0\ifxetex 1\fi\ifluatex 1\fi=0 % if pdftex
  \usepackage[T1]{fontenc}
  \usepackage[utf8]{inputenc}
\else % if luatex or xelatex
  \ifxetex
    \usepackage{mathspec}
  \else
    \usepackage{fontspec}
  \fi
  \defaultfontfeatures{Ligatures=TeX,Scale=MatchLowercase}
\fi
% use upquote if available, for straight quotes in verbatim environments
\IfFileExists{upquote.sty}{\usepackage{upquote}}{}
% use microtype if available
\IfFileExists{microtype.sty}{%
\usepackage[]{microtype}
\UseMicrotypeSet[protrusion]{basicmath} % disable protrusion for tt fonts
}{}
\PassOptionsToPackage{hyphens}{url} % url is loaded by hyperref
\usepackage[unicode=true]{hyperref}
\hypersetup{
            pdftitle={Kodkollektivet stadgar},
            pdfborder={0 0 0},
            breaklinks=true}
\urlstyle{same}  % don't use monospace font for urls
\IfFileExists{parskip.sty}{%
\usepackage{parskip}
}{% else
\setlength{\parindent}{0pt}
\setlength{\parskip}{6pt plus 2pt minus 1pt}
}
\setlength{\emergencystretch}{3em}  % prevent overfull lines
\providecommand{\tightlist}{%
  \setlength{\itemsep}{0pt}\setlength{\parskip}{0pt}}
\setcounter{secnumdepth}{0}
% Redefines (sub)paragraphs to behave more like sections
\ifx\paragraph\undefined\else
\let\oldparagraph\paragraph
\renewcommand{\paragraph}[1]{\oldparagraph{#1}\mbox{}}
\fi
\ifx\subparagraph\undefined\else
\let\oldsubparagraph\subparagraph
\renewcommand{\subparagraph}[1]{\oldsubparagraph{#1}\mbox{}}
\fi

% set default figure placement to htbp
\makeatletter
\def\fps@figure{htbp}
\makeatother


\title{Kodkollektivet stadgar}
\date{}

\begin{document}
\maketitle

\section{§1.1 Namn}\label{namn}

Föreningens namn är Kodkollektivet. Nedan kallas ``Föreningen''.

\section{§1.2 Säte}\label{suxe4te}

Föreningens säte är Växjö.

\section{§1.3 Syfte}\label{syfte}

Föreningens syfte är att främja och inspirera intresset för
programmering, mjukvaruutveckling, säkerhet och relaterade ämnen för
studenter inom Linnéuniversitetet.

\section{§1.4 Oberoende och
organisation}\label{oberoende-och-organisation}

Föreningen drivs ideellt, är partipolitiskt och religiöst oberoende.
Föreningens åsikter och ställningstaganden är inte heller representativa
för föreningens samarbetspartners eller vise versa.

\section{§1.5 Upplösning}\label{uppluxf6sning}

För upplösning av föreningen krävs godkännande av minst 90\% majoritet
av de röstande på nästkommande stormöte. Vid upplösning ska, efter att
eventuella skulder betalats, kvarvarande likvida medel tillfalla en
ideell organisation vald av föreningens medlemmar.

\section{§2.1 Medlemskap}\label{medlemskap}

En person äger rätt till medlemskap så länge hen är registrerad som
student vid Linnéuniversitetet och betalt medlemskap för terminen.

\section{§2.2 Medlemsavgift}\label{medlemsavgift}

Medlemsavgift erläggs av föreningens nyvarande styrelse på föreningens
Stormöte §3.2.

\section{§2.3 Medlemmar}\label{medlemmar}

Medlemskap erhålls för en termin genom inskrivning i medlemsregistret
och betalning av fastställd medlemsavgift.

\section{§2.3.1 Undantag}\label{undantag}

Sittande styrelsemedlemmar samt revisorn betalar ingen medlemsavgift men
är ändå medlemmar i föreningen.

\section{§2.4 Uteslutande av medlem}\label{uteslutande-av-medlem}

Styrelsen kan lägga fram en proposition om att utesluta en medlem om
denne har motarbetat förenings arbete, skadat föreningens verksamhet
eller anseende. Beslut om uteslutande bestäms på nästkommande stormöte
genom två tredjedelars majoritet av stormötets närvarande medlemmar.

\section{§3.1 Verksamhetsår}\label{verksamhetsuxe5r}

Föreningens verksamhetsår sträcker sig mellan 1 September till och med
31 Augusti nästföljande kalenderår.

\section{§3.2 Stormöte}\label{stormuxf6te}

Ett stormöte är föreningens högsta beslutande organ där samtliga
medlemmar har rösträtt. En kallelse till stormöte, samt en preliminär
föredragningslista §3.2.1, anslås på föreningens hemsida minst två (2)
veckor innan utsatt datum.

\section{§3.2.1 Föredragningslista}\label{fuxf6redragningslista}

Den slutgiltiga föredragningslistan måste göras tillgänglig senast en
(1) vecka innan utsatt datum för stormötet. Efter detta får
föredragningslistan ej ändras innan mötet. Varje föredragningslista
måste innehålla minst följande punkter:

\begin{itemize}
\tightlist
\item
  Val av mötesordförande
\item
  Val av mötessekreterare
\item
  Val av justeringsperson tillika rösträknare
\item
  Fastställande av röstlängden
\item
  Beslut om mötets stadgeenliga utlysande
\item
  Motioner och propositioner
\item
  Övriga frågor
\end{itemize}

\section{§3.2.2 Motioner}\label{motioner}

Motioner måste skickas in skriftligen till den sittande styrelsen senast
två (2) veckor innan utsatt datum för ett stormöte. De blir sedan en del
av den slutgiltiga föredragningslistan tillsammans med ett motionssvar
från den sittande styrelsen.

\section{§3.2.3 Beslut}\label{beslut}

Beslut fattas med enkel majoritet. Röstning med fullmakt får ej
förekomma. Medlemmar kan begära votering. Vid lika röstetal har
föreningens ordförande utslagsröst. För att föra in ett nytt ärende på
föredragningslistan erfordras 75\% majoritet. Under punkten ``övriga
frågor'' får det ej behandlas frågor som gäller kostnader.

\section{§3.2.4 Adjungering}\label{adjungering}

Stormötet kan adjungera personer. Med adjungering avses närvaro-,
yttrande- och förslagsrätt. Adjungering medför ej rätt att deltaga i
beslut, ej heller medansvar för fattade beslut.

\section{§3.2.5 Protokoll}\label{protokoll}

Stormöten måste protokollföras. Dessa skall anslås på föreningens
hemsida och skall arkiveras. Protokoll ska vara färdigställda inom fyra
(4) veckor efter ett möte.

\section{§3.2.6 Justering av protokoll}\label{justering-av-protokoll}

Protokoll från stormöte skall justeras av mötesordföranden,
mötessekreteraren och en av mötet utsedd justeringsperson.

\section{§3.3 Terminsmöte}\label{terminsmuxf6te}

Terminsmöte är ett stormöte som måste hållas i början på varje termin. Terminsmötets föredragningslista måste, förutom de som nämns i §3.2.1, minst lyfta punkterna:

\begin{itemize}
\tightlist
\item
  Fastställandet av nästa verksamhetsårs medlemsavgift
\item
  Val av nästföljande verksamhetsårs styrelse
\item
  Val av nästföljande verksamhetsårs revisor
\item
  Fastställande av nästkommande verksamhetsårs budget
\item
  Föregående verksamhetsårs styrelses verksamhetsberättelse
\item
  Föregående verksamhetsårs styrelses ekonomiska berättelse
\item
  Revisorns granskning av föregående verksamhetsårs styrelses arbete
\item
  Beslut om ansvarsfrihet av föregående verksamhetsårs styrelse
\end{itemize}

\section{§3.3.1 Extra stormöte}\label{extra-stormuxf6te}

Vid behov kan ett extra stormöte sammankallas av 50\% av
styrelseledamöterna. Vid yrkande om extra stormöte skall en kallelse
fastslås inom två (2) veckor innan mötet.

\section{§3.4 Styrdokument}\label{styrdokument}

Föreningens verksamhet regleras av dessa stadgar. För att ändra i
stadgarna krävs 75\% majoritet på ett stormöte.

\section{§3.4.1 Tolkningsfrågor}\label{tolkningsfruxe5gor}

Om tolkningsfrågor skulle uppstå i styrdokumentet gäller styrelsens
mening, tills frågan avgjorts på stormöte. Efter avklarad tolkningsfråga
skall formuleringen som gav upphov till situationen justeras enligt
stormötets beslut.

\section{§3.5 Entledigande}\label{entledigande}

Då särskilda skäl föreligger kan styrelsen efter skriftlig ansökan från
styrelsemedlem entlediga vederbörande samt tillförordna annan person att
fullgöra den entledigades uppgifter till nästa stormöte, då val skall
ske. Styrelsen äger ej rätt att entlediga:

\begin{itemize}
\tightlist
\item
  Ordförande
\item
  Kassör
\item
  Revisor
\end{itemize}

\section{§4 Styrelsen}\label{styrelsen}

Styrelsen handhar ledning av föreningens verksamhet i enlighet med
syftet, se §1.3, under verksamhetsåret. Styrelsen består minst av
följande ordinarie ledamöter:

\begin{itemize}
\tightlist
\item
  Ordförande
\item
  Vice Ordförande
\item
  Kassör
\end{itemize}

\section{§4.1 Rättigheter och
skyldigheter}\label{ruxe4ttigheter-och-skyldigheter}

Det åligger styrelsen att:

\begin{itemize}
\tightlist
\item
  Besluta om den löpande verksamheten
\item
  Bereda ärenden, vilka skall behandlas vid stormöten
\item
  Upprätta förslag till föredragningslista för stormöten
\item
  Inför stormöten ansvara för föreningens verksamhet
\item
  Verkställa av stormöten fattade beslut
\item
  Förbereda sina efterträdare inför deras verksamhetsår
\end{itemize}

\section{§4.2 Styrelsemöten}\label{styrelsemuxf6ten}

Styrelsemöten måste hållas minst en gång per termin under
verksamhetsåret. Styrelsemöten är beslutsmässiga då minst hälften av
ledamöterna är närvarande.

\section{§4.2.1 Adjungering}\label{adjungering-1}

Styrelsemötena likt stormötena kan adjungera personer, se §3.2.7

\section{§4.3 Protokoll}\label{protokoll-1}

Styrelsemöten måste protokollföras. Dessa skall anslås på föreningens
hemsida och skall arkiveras. Protokoll ska vara färdigställda inom fyra
(4) veckor efter ett möte.

\section{§4.4 Firmateckning}\label{firmateckning}

Föreningens firma, om sådan finns, tecknas av Ordföranden och Kassören
var för sig.

\section{§5 Revision}\label{revision}

En revisor skall väljas på terminsmötet som ska granska föreningens
verksamhet. Revisorn ska agera både sak- och sifferrevisor. Revisorn
skall vara myndig och får ej vara jävig.

\section{§5.1 Åligganden}\label{uxe5ligganden}

Revisorn skall före terminsmötet avsluta sin granskning av föregående
termins verksamhet och över den företagna revisionen upprätta
revisionsberättelse.

\section{§5.2 Handlingar}\label{handlingar}

Räkenskaper och övriga handlingar skall tillställas revisorn löpande
fram till terminsmötet.

\end{document}
